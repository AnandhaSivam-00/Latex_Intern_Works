\documentclass[12pt a4paper]{article}
\usepackage{graphicx} % Required for inserting images
\usepackage{ulem} %underline a text
\usepackage{draftwatermark} %watermark
\usepackage{soul} %for highlight
\usepackage{tikz} % For border
\usepackage[left=25mm, right=25mm, top=25mm, bottom=25mm, paper=a4paper]{geometry} % For page margins
\usetikzlibrary{shapes.geometric, positioning, calc}
\usepackage{setspace} % For spacing between the lines
\usepackage{subfigure} % Single line figures
\usepackage{wrapfig} % Multi figs
\usepackage{subfig}
\usepackage{enumitem}  % Ordered points
\usepackage{pifont}  % For Times Font
\usepackage{array}  % Table cell spacing
% \usepackage[table]{xcolor}
\usepackage{amsmath}  % Math mode
\usepackage{multirow}  % Column and row merging
\usepackage{fancyhdr}  % For header and footer
\usepackage{colortbl}  % Table cell colouring
\usepackage{ragged2e} % For centering

\renewcommand{\familydefault}{\rmdefault}
\spacing{2}


\SetWatermarkAngle{39}
\SetWatermarkScale{4}
\SetWatermarkText{Learn Basics}


\pagestyle{fancy}
\fancyhf{}
\rhead{ \fancyplain{} \Large \textcolor{gray}{3D Shapes}}

\renewcommand{\headrulewidth}{0pt}



\begin{document}

\begin{center}
\begin{Huge}
\textbf{\uline{Task 1}}
\end{Huge}
\thispagestyle{empty}
\end{center}


\newpage

\begin{tikzpicture}
[remember picture, overlay] \draw[line width=1pt]
($(current page.north west) + (0.6in, -0.6in)$) rectangle 
($(current page.south east) + (-0.6in, 0.6in)$);
\end{tikzpicture}
    
\begin{center}
\vspace*{\stretch{1}}
\begin{Huge}
\sethlcolor{yellow}
\hl{\textbf{3D Shapes}}
\end{Huge}
\vspace*{\stretch{1}}
\end{center}

\thispagestyle{empty}

\newpage
\clearpage


\begin{Huge}
\noindent
\textbf{3D Shapes} \\
\end{Huge}

\begin{huge}
\noindent \LARGE
3D shapes are solids that consist of \textbf{3 dimensions - length, breadth (width), and height}. 3D in the word 3D shapes means \textcolor{blue}{three-dimensional}. Every 3D geometric shape occupies some space based on its dimensions and we can see many 3D shapes all around us in our day-to-day life. Some examples of 3D shapes are \textbf{\textcolor{red}{cube, cuboid, cone, and cylinder}.}
\end{huge} \\

\begin{huge}
\noindent
\Large\textbf{Real-Life Examples of 3D Geometric Shapes}\\
\end{huge} \\

\begin{figure}[h]
    \centering
    \begin{subfigure}
    \centering
    \includegraphics[width=3cm]{football.jpg}
    \end{subfigure}
    \hspace{5em}
    \begin{subfigure}
    \centering
    \includegraphics[width=3cm]{rubics.png}
    \label{Figs}
    \end{subfigure}
\end{figure}
\begin{figure}[h]
    \centering
    \begin{subfigure}
    \centering
    \includegraphics[width=2cm]{bucket.png}
    \end{subfigure}
    \hspace{5em}
    \begin{subfigure}
    \centering
    \includegraphics[width=3cm]{book.png}
    \label{Figs1}
    \end{subfigure}
\end{figure}


\begin{huge}
\noindent
\Large\textbf{\uline{Types of 3D Shapes}}\\
\end{huge}

\begin{huge}
\noindent \LARGE
There are various types of 3D shapes that have different bases, volumes, and surface areas. \\ Let us discuss each one of them.
\end{huge} \\

\newpage

\noindent
\LARGE \textbf{Shapes}

\spacing{1.5}
\begin{enumerate}[label=\arabic*.]
    \item It is shaped like a ball and is perfectly symmetrical.
    \item It has a \textcolor{blue}{radius}, \textcolor{blue}{diameter}, circumference, volume, and surface area.
    \item Every point on the sphere is at an equal distance from the center.
    \item It has one face, no edges, and no vertices.
    \item It is not a polyhedron since it does not have flat faces.
\end{enumerate}

\noindent
\LARGE \textbf{Cylinder}
\begin{enumerate}[label=\alph*.]
    \item It has one curved face.
    \item The shape stays the same from the base to the top.
    \item It is a three-dimensional object with two identical ends that are either circular or \textcolor{blue}{oval}.
    \item A cylinder in which both circular bases lie on the same line is called a right cylinder. A cylinder in which one base is placed away from another is called an oblique cylinder.
\end{enumerate}


\newpage

\noindent
\LARGE \textbf{Cone}
\begin{itemize}
    \item[\ding{110}] A cone has a circular or oval base with an apex (vertex).
    \item[\ding{110}] A cone is a rotated triangle.
    \item[\ding{110}] Based on how the apex is aligned to the center of the base, a right cone or an oblique cone is formed.
\end{itemize}


\begin{figure}[h]
    \centering
    \includegraphics[width=0.75\linewidth]{Cone math.png}
    \caption{math}
    \label{mathe.png}
\end{figure}

\newpage

\spacing{1.5}
\begin{huge}
\noindent
\LARGE \textbf{Properties of 3D Shapes} \\
Every 3D shape has some properties which help us to 
identify them easily. Let us discuss each of them briefly. \\

\noindent

\centering
\renewcommand{\arraystretch}{1.6}
\begin{tabular}{|>{\centering\arraybackslash} m{3cm} | p{12.1cm} |}
    \hline
    \rowcolor{yellow}\textbf{3D Shapes} & \Centering{\textbf{Properties}} \\
    \hline
        \multirow{3}{*}{Cylinder} & 
        i) It has a flat base and a flat top. \\
        & ii) The bases are always congruent and parallel. \\ 
        & iii) It has one curved side. \\
    \hline
        \multirow{3}{*}{Cone}&
        iv) It has a flat base. \\ 
        & v) It has one curved side and one-pointed vertex at the top or bottom known as the apex. \\
    \hline
        \multirow{3}{*}{Cube}&
        vi) It has six faces in the shape of a square. \\
        & vii) The sides are of equal lengths. \\
        & viii) 12 diagonals can be drawn on a cube. \\
    \hline
\end{tabular}


\end{huge}

\newpage


\noindent
\LARGE \textbf{3D Shapes Formulas} \\
As discussed, all 3 Dimensional shapes have a surface area and volume. The following table shows different 3D shapes and their formulas. \\

\spacing{1.4}
\renewcommand{\arraystretch}{1.4}
\begin{tabular}{|>{\centering\arraybackslash} m{3cm} |>{\centering\arraybackslash} p{3.5cm} |>{\centering\arraybackslash} p{7.5cm} | }
    \hline
    \rowcolor{yellow}
    \textbf{3D Shape} & 
    \multicolumn{2}{|c|}{\textbf{Formulas}} 
    \cr
    \hline
    \vspace{0.5cm} \multirow{3}{*}{Sphere} & Diameter  & $2 \times r$; \newline (where 'r' is the radius) \\
    \cline{2-3}
           & Surface Area & $4\pi r^2$ \\
    \cline{2-3} 
           & Volume & $\frac{4}{3}\pi r^3$ \\
    \hline
    \vspace{1.1cm} \multirow{2}{*}{Cylinder} & Total Surface Area & $2\pi (r+h+r)$; \newline
              (where 'r' is the radius and 'h' is the height of the cylinder) \\ 
    \cline{2-3}
              & Volume & $\pi r^2 h$ \\
    \hline
    \vspace{1cm} \multirow{2}{*}{Cone} & Curved Surface Area & $\pi rl$; \newline (where 'l' is the slant height and  l = $\sqrt{(h^2 + r^2)}$) \\ 
    \cline{2-3}
              & Volume & $\frac{\pi}{3}r^2 h$ \\
    \hline
\end{tabular}



\end{document}
