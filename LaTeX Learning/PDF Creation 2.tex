\documentclass[12pt a4paper]{article}
\usepackage{graphicx} % Required for inserting images
\usepackage{ulem} %underline a text
\usepackage{draftwatermark} %watermark
\usepackage{soul} %for highlight
\usepackage{tikz} % For border
\usepackage[left=25mm, right=25mm, top=25mm, bottom=25mm, paper=a4paper]{geometry} % For page margins
\usetikzlibrary{shapes.geometric, positioning, calc}
\usepackage{setspace} % For spacing between the lines
\usepackage{fancyhdr}  % For header and footer
\usepackage{moresize} % For more sizing
\usepackage{ragged2e} % For centering
\usepackage{wrapfig}
\usepackage{pifont}  % For Times Font
\usepackage[many]{tcolorbox}
\usepackage{xcolor}
\usepackage{adjustbox}
\usepackage{multicol} % for two column in one page


\renewcommand{\familydefault}{\rmdefault}
\spacing{2}

\newtcolorbox{dashedbox}{
    colback = white, 
    enhanced,
    boxrule = 1.5pt, % for padding
    colframe = white, % making the base for dash line
    borderline = {1.7pt}{0pt}{blue, dotted} % add "dashed" for dashed line
}


\SetWatermarkAngle{39}
\SetWatermarkScale{4}
\SetWatermarkText{Learn Basics}


\pagestyle{fancy}
\fancyhf{}
\rhead{ \fancyplain{} \Large \textcolor{gray}{3D Shapes}}
\fancyfoot[C]{\thepage}

\renewcommand{\headrulewidth}{0pt}

\begin{document}

\begin{center}

\HUGE \textbf{\uline{Task 2}}

\thispagestyle{empty}
\end{center}


\newpage
\thispagestyle{empty}

\begin{tikzpicture}
[remember picture, overlay] \draw[line width=1pt]
($(current page.north west) + (0.6in, -0.6in)$) rectangle 
($(current page.south east) + (-0.6in, 0.6in)$);
\end{tikzpicture}
    
\spacing{3}
\vspace*{\stretch{1}}
\begin{Huge}
\sethlcolor{yellow}
\RaggedLeft{\hl{\textbf{MATHEMATICS STUDY \\  SKILLS GUIDE}}}
\end{Huge}
\vspace*{\stretch{1}}

\newpage

\spacing{1.5}
\begin{huge}
\noindent \LARGE
Learning mathematics is different than learning most other subjects. In mathematics, special vocabulary and symbols are used and it is important that you not only understand the concepts being presented, but that 
you also apply these concepts. To be successful in mathematics, you need not only to read, attend class, and study, but you must practice the skills as often as you can. Mathematics is not a subject you learn by watching; you must DO mathematics to LEARN mathematics. \\
\begin{wrapfigure}{r}{0.7\textwidth}
    \centering
    \includegraphics[width=12cm, height=8cm]{matematicaS2.jpg}
    \label{fig: Maths}
\end{wrapfigure}
The purpose of this STUDY SKILLS GUIDE is to present you with strategies for studying that have been effective for students in mathematics classes. Developing good study habits is one of the keys to being a successful learner of mathematics. Nine strategies are described in this guide.

\end{huge}

\newpage

\spacing{1.5}
\indent \LARGE 
\begin{wrapfigure}{l}{0.4\textwidth}
    \centering
    \includegraphics[width=6cm, height=7cm]{School_Board.jpg}
    \label{fig: Board}
\end{wrapfigure}
Of course, attending class and paying close attention and taking good notes in class is also very important. Combining your classroom learning and your own studying, you can be a successful learner of mathematics. \\
\newline

\spacing{1.5}
\begin{dashedbox}
\vspace{0.3cm}
\LARGE \textbf{\uline{Students' advice for being successful}} \\
\LARGE I found that doing the homework day-by-day (a little every night) really helped me. Make sure you do all the homework.
The flash cards were helpful. \\
Going over and over homework as well as notes. With no homework a student won’t survive. Keep up. If falling behind, get help. \\
Don’t miss any classes. Attendance is critical. I struggled the week or two that I missed a class.
\vspace{0.3cm}
\end{dashedbox}

\newpage

\spacing{1.5}
\begin{dashedbox}
\vspace{0.3cm}
\LARGE \textbf{\uline{A note about mathematics vocabulary }} \\
\LARGE Some words used in mathematics are not used outside of the subject. But many mathematical terms are used elsewhere in everyday language. \\ Distinguishing different meanings of a word--and its special meaning in mathematics--is an important part of learning to do mathematics. \vspace{0.3cm} \\
Examples: \\
“Power" refers to an exponent in mathematics, but has many other meanings, such as in electrical power, in other settings. 
\vspace{0.3cm} \\
“Difference” is a mathematical term that indicates the result of the operation of subtraction. In everyday usage, “difference” refers to how two or more things are not alike. 
\vspace{0.3cm} \\
\Centering{\includegraphics[width=0.7\linewidth]{Math Homepage.jpg}} \\
\vspace{0.3cm}

\end{dashedbox}

\newpage


\noindent \LARGE
As in the previous study strategy, your instructor may ask that you NOT use your textbook as a study aid. In these cases, you will need to rely on your notes as you do the homework exercises.
\begin{wrapfigure}{l}{0.5\textwidth}
    \centering
    \includegraphics[width=0.7\linewidth]{studying.png}
    \label{fig: Maths}
\end{wrapfigure} \newline
Be sure, in any case, that you are taking \textbf{thorough notes} in class. Be sure, in any case, that you are taking \textbf{thorough notes} in class. Do not just do the exercises at the beginning of the problem set. Usually, the exercises get harder as you move on.
\\


\spacing{1.5}
\begin{dashedbox}
\vspace{0.3cm}
\LARGE \textbf{\uline{A note about doing homework}} \vspace{0.5cm} \\
\LARGE Do not just do the exercises at the beginning of the problem set. Usually the exercises get harder as you move on. It is best to do some of each--from simpler to harder--at first, then go back and do the ones you skipped. Make notes to yourself as you do your homework, especially on concepts that are not completely clear to you. You can ask about those problems during your next class meeting.
\vspace{0.3cm}
\end{dashedbox}

\newpage

\spacing{1.3}
\setlength{\columnsep}{0.8cm}
\begin{multicols*}{2}
\noindent
When doing your homework (including the examples), use your notes as a guide
and write the procedures you use in completing an exercise. Often, there are different procedures used in problems involving the same concept, such as
solving an equation. \\
\newline
    \includegraphics[width=0.9\linewidth]{grade.png} \\
\newline
The more you write what you are doing, the better you will remember it. Once you are comfortable doing a procedure, it is not necessary to write it each time you do it. Pay special attention to the directions for completing the exercises. You will need to know what the directions mean for you to do and
what procedures are used to carry out the directions.\\ \newline
The directions often use the specialized vocabulary of mathematics, so it is important to recognize the \textbf{key terms} (such as “simplify” or “solve”). Also, the directions to exercises may ask you to do something that is different than what you would expect to do. (For example, an equation may be given, and you may be asked to tell what type of equation it is, not to solve it. 
\end{multicols*}




\end{document}